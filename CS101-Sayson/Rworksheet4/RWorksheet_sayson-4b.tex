% Options for packages loaded elsewhere
\PassOptionsToPackage{unicode}{hyperref}
\PassOptionsToPackage{hyphens}{url}
%
\documentclass[
]{article}
\usepackage{amsmath,amssymb}
\usepackage{iftex}
\ifPDFTeX
  \usepackage[T1]{fontenc}
  \usepackage[utf8]{inputenc}
  \usepackage{textcomp} % provide euro and other symbols
\else % if luatex or xetex
  \usepackage{unicode-math} % this also loads fontspec
  \defaultfontfeatures{Scale=MatchLowercase}
  \defaultfontfeatures[\rmfamily]{Ligatures=TeX,Scale=1}
\fi
\usepackage{lmodern}
\ifPDFTeX\else
  % xetex/luatex font selection
\fi
% Use upquote if available, for straight quotes in verbatim environments
\IfFileExists{upquote.sty}{\usepackage{upquote}}{}
\IfFileExists{microtype.sty}{% use microtype if available
  \usepackage[]{microtype}
  \UseMicrotypeSet[protrusion]{basicmath} % disable protrusion for tt fonts
}{}
\makeatletter
\@ifundefined{KOMAClassName}{% if non-KOMA class
  \IfFileExists{parskip.sty}{%
    \usepackage{parskip}
  }{% else
    \setlength{\parindent}{0pt}
    \setlength{\parskip}{6pt plus 2pt minus 1pt}}
}{% if KOMA class
  \KOMAoptions{parskip=half}}
\makeatother
\usepackage{xcolor}
\usepackage[margin=1in]{geometry}
\usepackage{color}
\usepackage{fancyvrb}
\newcommand{\VerbBar}{|}
\newcommand{\VERB}{\Verb[commandchars=\\\{\}]}
\DefineVerbatimEnvironment{Highlighting}{Verbatim}{commandchars=\\\{\}}
% Add ',fontsize=\small' for more characters per line
\usepackage{framed}
\definecolor{shadecolor}{RGB}{248,248,248}
\newenvironment{Shaded}{\begin{snugshade}}{\end{snugshade}}
\newcommand{\AlertTok}[1]{\textcolor[rgb]{0.94,0.16,0.16}{#1}}
\newcommand{\AnnotationTok}[1]{\textcolor[rgb]{0.56,0.35,0.01}{\textbf{\textit{#1}}}}
\newcommand{\AttributeTok}[1]{\textcolor[rgb]{0.13,0.29,0.53}{#1}}
\newcommand{\BaseNTok}[1]{\textcolor[rgb]{0.00,0.00,0.81}{#1}}
\newcommand{\BuiltInTok}[1]{#1}
\newcommand{\CharTok}[1]{\textcolor[rgb]{0.31,0.60,0.02}{#1}}
\newcommand{\CommentTok}[1]{\textcolor[rgb]{0.56,0.35,0.01}{\textit{#1}}}
\newcommand{\CommentVarTok}[1]{\textcolor[rgb]{0.56,0.35,0.01}{\textbf{\textit{#1}}}}
\newcommand{\ConstantTok}[1]{\textcolor[rgb]{0.56,0.35,0.01}{#1}}
\newcommand{\ControlFlowTok}[1]{\textcolor[rgb]{0.13,0.29,0.53}{\textbf{#1}}}
\newcommand{\DataTypeTok}[1]{\textcolor[rgb]{0.13,0.29,0.53}{#1}}
\newcommand{\DecValTok}[1]{\textcolor[rgb]{0.00,0.00,0.81}{#1}}
\newcommand{\DocumentationTok}[1]{\textcolor[rgb]{0.56,0.35,0.01}{\textbf{\textit{#1}}}}
\newcommand{\ErrorTok}[1]{\textcolor[rgb]{0.64,0.00,0.00}{\textbf{#1}}}
\newcommand{\ExtensionTok}[1]{#1}
\newcommand{\FloatTok}[1]{\textcolor[rgb]{0.00,0.00,0.81}{#1}}
\newcommand{\FunctionTok}[1]{\textcolor[rgb]{0.13,0.29,0.53}{\textbf{#1}}}
\newcommand{\ImportTok}[1]{#1}
\newcommand{\InformationTok}[1]{\textcolor[rgb]{0.56,0.35,0.01}{\textbf{\textit{#1}}}}
\newcommand{\KeywordTok}[1]{\textcolor[rgb]{0.13,0.29,0.53}{\textbf{#1}}}
\newcommand{\NormalTok}[1]{#1}
\newcommand{\OperatorTok}[1]{\textcolor[rgb]{0.81,0.36,0.00}{\textbf{#1}}}
\newcommand{\OtherTok}[1]{\textcolor[rgb]{0.56,0.35,0.01}{#1}}
\newcommand{\PreprocessorTok}[1]{\textcolor[rgb]{0.56,0.35,0.01}{\textit{#1}}}
\newcommand{\RegionMarkerTok}[1]{#1}
\newcommand{\SpecialCharTok}[1]{\textcolor[rgb]{0.81,0.36,0.00}{\textbf{#1}}}
\newcommand{\SpecialStringTok}[1]{\textcolor[rgb]{0.31,0.60,0.02}{#1}}
\newcommand{\StringTok}[1]{\textcolor[rgb]{0.31,0.60,0.02}{#1}}
\newcommand{\VariableTok}[1]{\textcolor[rgb]{0.00,0.00,0.00}{#1}}
\newcommand{\VerbatimStringTok}[1]{\textcolor[rgb]{0.31,0.60,0.02}{#1}}
\newcommand{\WarningTok}[1]{\textcolor[rgb]{0.56,0.35,0.01}{\textbf{\textit{#1}}}}
\usepackage{graphicx}
\makeatletter
\newsavebox\pandoc@box
\newcommand*\pandocbounded[1]{% scales image to fit in text height/width
  \sbox\pandoc@box{#1}%
  \Gscale@div\@tempa{\textheight}{\dimexpr\ht\pandoc@box+\dp\pandoc@box\relax}%
  \Gscale@div\@tempb{\linewidth}{\wd\pandoc@box}%
  \ifdim\@tempb\p@<\@tempa\p@\let\@tempa\@tempb\fi% select the smaller of both
  \ifdim\@tempa\p@<\p@\scalebox{\@tempa}{\usebox\pandoc@box}%
  \else\usebox{\pandoc@box}%
  \fi%
}
% Set default figure placement to htbp
\def\fps@figure{htbp}
\makeatother
\setlength{\emergencystretch}{3em} % prevent overfull lines
\providecommand{\tightlist}{%
  \setlength{\itemsep}{0pt}\setlength{\parskip}{0pt}}
\setcounter{secnumdepth}{-\maxdimen} % remove section numbering
\usepackage{bookmark}
\IfFileExists{xurl.sty}{\usepackage{xurl}}{} % add URL line breaks if available
\urlstyle{same}
\hypersetup{
  pdftitle={RWorksheet\_sayson\#4b.Rmd},
  pdfauthor={Adrian T. Sayson},
  hidelinks,
  pdfcreator={LaTeX via pandoc}}

\title{RWorksheet\_sayson\#4b.Rmd}
\author{Adrian T. Sayson}
\date{2025-11-24}

\begin{document}
\maketitle

\begin{Shaded}
\begin{Highlighting}[]
\CommentTok{\#1.}
\NormalTok{vectorA }\OtherTok{\textless{}{-}} \FunctionTok{c}\NormalTok{(}\DecValTok{1}\NormalTok{, }\DecValTok{2}\NormalTok{, }\DecValTok{3}\NormalTok{, }\DecValTok{4}\NormalTok{, }\DecValTok{5}\NormalTok{)}
\NormalTok{matrix5x5 }\OtherTok{\textless{}{-}} \FunctionTok{matrix}\NormalTok{(}\DecValTok{0}\NormalTok{, }\AttributeTok{nrow =} \DecValTok{5}\NormalTok{, }\AttributeTok{ncol =} \DecValTok{5}\NormalTok{)}

\ControlFlowTok{for}\NormalTok{(i }\ControlFlowTok{in} \DecValTok{1}\SpecialCharTok{:}\DecValTok{5}\NormalTok{) \{      }
  \ControlFlowTok{for}\NormalTok{(j }\ControlFlowTok{in} \DecValTok{1}\SpecialCharTok{:}\DecValTok{5}\NormalTok{) \{      }
\NormalTok{    matrix5x5[i, j] }\OtherTok{\textless{}{-}} \FunctionTok{abs}\NormalTok{(vectorA[j] }\SpecialCharTok{{-}}\NormalTok{ i)}
\NormalTok{  \}}
\NormalTok{\}}
\FunctionTok{print}\NormalTok{(matrix5x5)}
\end{Highlighting}
\end{Shaded}

\begin{verbatim}
##      [,1] [,2] [,3] [,4] [,5]
## [1,]    0    1    2    3    4
## [2,]    1    0    1    2    3
## [3,]    2    1    0    1    2
## [4,]    3    2    1    0    1
## [5,]    4    3    2    1    0
\end{verbatim}

\begin{Shaded}
\begin{Highlighting}[]
\CommentTok{\#2.}
\ControlFlowTok{for}\NormalTok{(i }\ControlFlowTok{in} \DecValTok{1}\SpecialCharTok{:}\DecValTok{5}\NormalTok{)\{}
\ControlFlowTok{for}\NormalTok{(j }\ControlFlowTok{in} \DecValTok{1}\SpecialCharTok{:}\NormalTok{i)\{}
\FunctionTok{cat}\NormalTok{(}\StringTok{"*"}\NormalTok{)}
\NormalTok{\}}
\FunctionTok{cat}\NormalTok{(}\StringTok{"}\SpecialCharTok{\textbackslash{}n}\StringTok{"}\NormalTok{)}
\NormalTok{\}}
\end{Highlighting}
\end{Shaded}

\begin{verbatim}
## *
## **
## ***
## ****
## *****
\end{verbatim}

\begin{Shaded}
\begin{Highlighting}[]
\CommentTok{\#3.}
\NormalTok{a }\OtherTok{\textless{}{-}} \DecValTok{0}
\NormalTok{b }\OtherTok{\textless{}{-}} \DecValTok{1}

\FunctionTok{cat}\NormalTok{(a, }\StringTok{", "}\NormalTok{, }\AttributeTok{sep =} \StringTok{""}\NormalTok{)}
\end{Highlighting}
\end{Shaded}

\begin{verbatim}
## 0,
\end{verbatim}

\begin{Shaded}
\begin{Highlighting}[]
\ControlFlowTok{repeat}\NormalTok{ \{}
  \FunctionTok{cat}\NormalTok{(b, }\StringTok{", "}\NormalTok{, }\AttributeTok{sep =} \StringTok{""}\NormalTok{)}
\NormalTok{  next\_val }\OtherTok{\textless{}{-}}\NormalTok{ a }\SpecialCharTok{+}\NormalTok{ b}
\NormalTok{  a }\OtherTok{\textless{}{-}}\NormalTok{ b}
\NormalTok{  b }\OtherTok{\textless{}{-}}\NormalTok{ next\_val}
  
  \ControlFlowTok{if}\NormalTok{ (b }\SpecialCharTok{\textgreater{}} \DecValTok{500}\NormalTok{) \{}
    \ControlFlowTok{break}
\NormalTok{  \}}
\NormalTok{\}}
\end{Highlighting}
\end{Shaded}

\begin{verbatim}
## 1, 1, 2, 3, 5, 8, 13, 21, 34, 55, 89, 144, 233, 377,
\end{verbatim}

\begin{Shaded}
\begin{Highlighting}[]
\NormalTok{fib }\OtherTok{\textless{}{-}} \FunctionTok{numeric}\NormalTok{()}
\NormalTok{a }\OtherTok{\textless{}{-}} \DecValTok{0}
\NormalTok{b }\OtherTok{\textless{}{-}} \DecValTok{1}
\ControlFlowTok{while}\NormalTok{(b }\SpecialCharTok{\textless{}=} \DecValTok{500}\NormalTok{)\{}
\NormalTok{  fib }\OtherTok{\textless{}{-}} \FunctionTok{c}\NormalTok{(fib, b)}
\NormalTok{  next\_val }\OtherTok{\textless{}{-}}\NormalTok{ a }\SpecialCharTok{+}\NormalTok{ b}
\NormalTok{  a }\OtherTok{\textless{}{-}}\NormalTok{ b}
\NormalTok{  b }\OtherTok{\textless{}{-}}\NormalTok{ next\_val}
\NormalTok{\}}
\FunctionTok{cat}\NormalTok{(}\StringTok{"Fibonacci numbers up to 500:"}\NormalTok{, }\FunctionTok{paste}\NormalTok{(fib, }\AttributeTok{collapse =} \StringTok{", "}\NormalTok{))}
\end{Highlighting}
\end{Shaded}

\begin{verbatim}
## Fibonacci numbers up to 500: 1, 1, 2, 3, 5, 8, 13, 21, 34, 55, 89, 144, 233, 377
\end{verbatim}

\begin{Shaded}
\begin{Highlighting}[]
\CommentTok{\#4.}
\NormalTok{shoess }\OtherTok{\textless{}{-}} \FunctionTok{read.csv}\NormalTok{(}\StringTok{"shoe\_table.csv"}\NormalTok{)}
\FunctionTok{print}\NormalTok{(shoess)}
\end{Highlighting}
\end{Shaded}

\begin{verbatim}
##    Shoe_size Height Gender
## 1        6.5   66.0      F
## 2        9.0   68.0      F
## 3        8.5   64.5      F
## 4        8.5   65.0      F
## 5       10.5   70.0      M
## 6        7.0   64.0      F
## 7        9.5   70.0      F
## 8        9.0   71.0      F
## 9       13.0   72.0      M
## 10       7.5   64.0      F
## 11      10.5   74.5      M
## 12       8.5   67.0      F
## 13      12.0   71.0      M
## 14      10.5   71.0      M
## 15      13.0   77.0      M
## 16      11.5   72.0      M
## 17       8.5   59.0      F
## 18       5.0   62.0      F
## 19      10.0   72.0      M
## 20       6.5   66.0      F
## 21       7.5   64.0      F
## 22       8.5   67.0      M
## 23      10.5   73.0      M
## 24      10.5   72.0      M
## 25      11.0   69.0      M
## 26       9.0   69.0      M
## 27      13.0   70.0      M
\end{verbatim}

\begin{Shaded}
\begin{Highlighting}[]
\CommentTok{\#A}
\FunctionTok{head}\NormalTok{(shoess)}
\end{Highlighting}
\end{Shaded}

\begin{verbatim}
##   Shoe_size Height Gender
## 1       6.5   66.0      F
## 2       9.0   68.0      F
## 3       8.5   64.5      F
## 4       8.5   65.0      F
## 5      10.5   70.0      M
## 6       7.0   64.0      F
\end{verbatim}

\begin{Shaded}
\begin{Highlighting}[]
\CommentTok{\#b}
\NormalTok{female }\OtherTok{\textless{}{-}} \FunctionTok{subset}\NormalTok{(shoess, Gender }\SpecialCharTok{==} \StringTok{"F"}\NormalTok{)}
\NormalTok{male }\OtherTok{\textless{}{-}} \FunctionTok{subset}\NormalTok{(shoess, Gender }\SpecialCharTok{==} \StringTok{"M"}\NormalTok{)}
\FunctionTok{nrow}\NormalTok{(male)}
\end{Highlighting}
\end{Shaded}

\begin{verbatim}
## [1] 14
\end{verbatim}

\begin{Shaded}
\begin{Highlighting}[]
\FunctionTok{nrow}\NormalTok{(female)}
\end{Highlighting}
\end{Shaded}

\begin{verbatim}
## [1] 13
\end{verbatim}

\begin{Shaded}
\begin{Highlighting}[]
\CommentTok{\#c}
\NormalTok{gender\_ct }\OtherTok{\textless{}{-}} \FunctionTok{table}\NormalTok{(shoess}\SpecialCharTok{$}\NormalTok{Gender)}

\FunctionTok{barplot}\NormalTok{(gender\_ct,}
        \AttributeTok{main =} \StringTok{"Number of Males and Females"}\NormalTok{,}
        \AttributeTok{xlab =} \StringTok{"Gender"}\NormalTok{,}
        \AttributeTok{ylab =} \StringTok{"Count"}\NormalTok{,}
        \AttributeTok{col =} \FunctionTok{c}\NormalTok{(}\StringTok{"cyan"}\NormalTok{, }\StringTok{"pink"}\NormalTok{),}
        \AttributeTok{legend.text =} \ConstantTok{TRUE}\NormalTok{)}
\end{Highlighting}
\end{Shaded}

\pandocbounded{\includegraphics[keepaspectratio]{RWorksheet_sayson-4b_files/figure-latex/unnamed-chunk-8-1.pdf}}

\begin{Shaded}
\begin{Highlighting}[]
\CommentTok{\#5}
\NormalTok{expenses }\OtherTok{\textless{}{-}} \FunctionTok{c}\NormalTok{(}\DecValTok{60}\NormalTok{, }\DecValTok{10}\NormalTok{, }\DecValTok{5}\NormalTok{, }\DecValTok{25}\NormalTok{)}
\NormalTok{categories }\OtherTok{\textless{}{-}} \FunctionTok{c}\NormalTok{(}\StringTok{"Food"}\NormalTok{, }\StringTok{"Electricity"}\NormalTok{, }\StringTok{"Savings"}\NormalTok{, }\StringTok{"Miscellaneous"}\NormalTok{)}
\NormalTok{percent }\OtherTok{\textless{}{-}} \FunctionTok{round}\NormalTok{(expenses }\SpecialCharTok{/} \FunctionTok{sum}\NormalTok{(expenses) }\SpecialCharTok{*} \DecValTok{100}\NormalTok{)}
\NormalTok{labels }\OtherTok{\textless{}{-}} \FunctionTok{paste}\NormalTok{(categories, percent, }\StringTok{"\%"}\NormalTok{)}
\FunctionTok{pie}\NormalTok{(expenses,}
    \AttributeTok{labels =}\NormalTok{ labels,}
    \AttributeTok{col =} \FunctionTok{c}\NormalTok{(}\StringTok{"red"}\NormalTok{, }\StringTok{"yellow"}\NormalTok{, }\StringTok{"green"}\NormalTok{, }\StringTok{"cyan"}\NormalTok{),}
    \AttributeTok{main =} \StringTok{"Monthly Income Distribution of Dela Cruz Family"}\NormalTok{)}
\end{Highlighting}
\end{Shaded}

\pandocbounded{\includegraphics[keepaspectratio]{RWorksheet_sayson-4b_files/figure-latex/unnamed-chunk-9-1.pdf}}

\begin{Shaded}
\begin{Highlighting}[]
\CommentTok{\#a}
\FunctionTok{str}\NormalTok{(iris)}
\end{Highlighting}
\end{Shaded}

\begin{verbatim}
## 'data.frame':    150 obs. of  5 variables:
##  $ Sepal.Length: num  5.1 4.9 4.7 4.6 5 5.4 4.6 5 4.4 4.9 ...
##  $ Sepal.Width : num  3.5 3 3.2 3.1 3.6 3.9 3.4 3.4 2.9 3.1 ...
##  $ Petal.Length: num  1.4 1.4 1.3 1.5 1.4 1.7 1.4 1.5 1.4 1.5 ...
##  $ Petal.Width : num  0.2 0.2 0.2 0.2 0.2 0.4 0.3 0.2 0.2 0.1 ...
##  $ Species     : Factor w/ 3 levels "setosa","versicolor",..: 1 1 1 1 1 1 1 1 1 1 ...
\end{verbatim}

\begin{Shaded}
\begin{Highlighting}[]
\CommentTok{\#There is 150 rows and 5 columns. The dataset has 3 species of the "Iris" flowers which are then each classified with 4 measurements.}
\end{Highlighting}
\end{Shaded}

\begin{Shaded}
\begin{Highlighting}[]
\CommentTok{\#b}
\NormalTok{means\_iris }\OtherTok{\textless{}{-}} \FunctionTok{colMeans}\NormalTok{(iris[, }\DecValTok{1}\SpecialCharTok{:}\DecValTok{4}\NormalTok{])}

\NormalTok{means\_iris}
\end{Highlighting}
\end{Shaded}

\begin{verbatim}
## Sepal.Length  Sepal.Width Petal.Length  Petal.Width 
##     5.843333     3.057333     3.758000     1.199333
\end{verbatim}

\begin{Shaded}
\begin{Highlighting}[]
\CommentTok{\#c}
\NormalTok{species\_count }\OtherTok{\textless{}{-}} \FunctionTok{table}\NormalTok{(iris}\SpecialCharTok{$}\NormalTok{Species)}

\FunctionTok{pie}\NormalTok{(species\_count,}
    \AttributeTok{main =} \StringTok{"Distribution of Iris Species"}\NormalTok{,}
    \AttributeTok{col =} \FunctionTok{c}\NormalTok{(}\StringTok{"yellow"}\NormalTok{, }\StringTok{"red"}\NormalTok{, }\StringTok{"blue"}\NormalTok{),}
    \AttributeTok{labels =} \FunctionTok{paste}\NormalTok{(}\FunctionTok{names}\NormalTok{(species\_count), }\StringTok{"("}\NormalTok{, species\_count, }\StringTok{")"}\NormalTok{)}
\NormalTok{)}
\FunctionTok{legend}\NormalTok{(}\StringTok{"bottomleft"}\NormalTok{,}
       \AttributeTok{legend =} \FunctionTok{names}\NormalTok{(species\_count),}
       \AttributeTok{fill =} \FunctionTok{c}\NormalTok{(}\StringTok{"yellow"}\NormalTok{, }\StringTok{"red"}\NormalTok{, }\StringTok{"blue"}\NormalTok{),}
       \AttributeTok{title =} \StringTok{"Species"}\NormalTok{)}
\end{Highlighting}
\end{Shaded}

\pandocbounded{\includegraphics[keepaspectratio]{RWorksheet_sayson-4b_files/figure-latex/unnamed-chunk-12-1.pdf}}

\begin{Shaded}
\begin{Highlighting}[]
\CommentTok{\#d}
\NormalTok{setosa }\OtherTok{\textless{}{-}} \FunctionTok{subset}\NormalTok{(iris, Species }\SpecialCharTok{==} \StringTok{"setosa"}\NormalTok{)}
\NormalTok{versicolor }\OtherTok{\textless{}{-}} \FunctionTok{subset}\NormalTok{(iris, Species }\SpecialCharTok{==} \StringTok{"versicolor"}\NormalTok{)}
\NormalTok{virginica }\OtherTok{\textless{}{-}} \FunctionTok{subset}\NormalTok{(iris, Species }\SpecialCharTok{==} \StringTok{"virginica"}\NormalTok{)}

\FunctionTok{tail}\NormalTok{(setosa)}
\end{Highlighting}
\end{Shaded}

\begin{verbatim}
##    Sepal.Length Sepal.Width Petal.Length Petal.Width Species
## 45          5.1         3.8          1.9         0.4  setosa
## 46          4.8         3.0          1.4         0.3  setosa
## 47          5.1         3.8          1.6         0.2  setosa
## 48          4.6         3.2          1.4         0.2  setosa
## 49          5.3         3.7          1.5         0.2  setosa
## 50          5.0         3.3          1.4         0.2  setosa
\end{verbatim}

\begin{Shaded}
\begin{Highlighting}[]
\FunctionTok{tail}\NormalTok{(versicolor)}
\end{Highlighting}
\end{Shaded}

\begin{verbatim}
##     Sepal.Length Sepal.Width Petal.Length Petal.Width    Species
## 95           5.6         2.7          4.2         1.3 versicolor
## 96           5.7         3.0          4.2         1.2 versicolor
## 97           5.7         2.9          4.2         1.3 versicolor
## 98           6.2         2.9          4.3         1.3 versicolor
## 99           5.1         2.5          3.0         1.1 versicolor
## 100          5.7         2.8          4.1         1.3 versicolor
\end{verbatim}

\begin{Shaded}
\begin{Highlighting}[]
\FunctionTok{tail}\NormalTok{(virginica)}
\end{Highlighting}
\end{Shaded}

\begin{verbatim}
##     Sepal.Length Sepal.Width Petal.Length Petal.Width   Species
## 145          6.7         3.3          5.7         2.5 virginica
## 146          6.7         3.0          5.2         2.3 virginica
## 147          6.3         2.5          5.0         1.9 virginica
## 148          6.5         3.0          5.2         2.0 virginica
## 149          6.2         3.4          5.4         2.3 virginica
## 150          5.9         3.0          5.1         1.8 virginica
\end{verbatim}

\begin{Shaded}
\begin{Highlighting}[]
\CommentTok{\#e}
\NormalTok{iris}\SpecialCharTok{$}\NormalTok{Species }\OtherTok{\textless{}{-}} \FunctionTok{as.factor}\NormalTok{(iris}\SpecialCharTok{$}\NormalTok{Species)}

\NormalTok{colors }\OtherTok{\textless{}{-}} \FunctionTok{c}\NormalTok{(}\StringTok{"setosa"} \OtherTok{=} \StringTok{"orange"}\NormalTok{,}
            \StringTok{"versicolor"} \OtherTok{=} \StringTok{"purple"}\NormalTok{,}
            \StringTok{"virginica"} \OtherTok{=} \StringTok{"green"}\NormalTok{)}

\NormalTok{symbols }\OtherTok{\textless{}{-}} \FunctionTok{c}\NormalTok{(}\StringTok{"setosa"} \OtherTok{=} \DecValTok{16}\NormalTok{,    }
             \StringTok{"versicolor"} \OtherTok{=} \DecValTok{17}\NormalTok{,   }
             \StringTok{"virginica"} \OtherTok{=} \DecValTok{15}\NormalTok{)   }

\FunctionTok{plot}\NormalTok{(iris}\SpecialCharTok{$}\NormalTok{Sepal.Length, iris}\SpecialCharTok{$}\NormalTok{Sepal.Width,}
     \AttributeTok{col =}\NormalTok{ colors[iris}\SpecialCharTok{$}\NormalTok{Species],}
     \AttributeTok{pch =}\NormalTok{ symbols[iris}\SpecialCharTok{$}\NormalTok{Species],}
     \AttributeTok{main =} \StringTok{"Iris Dataset"}\NormalTok{,}
     \AttributeTok{sub =} \StringTok{"Sepal Width and Length"}\NormalTok{,}
     \AttributeTok{xlab =} \StringTok{"Sepal Length"}\NormalTok{,}
     \AttributeTok{ylab =} \StringTok{"Sepal Width"}\NormalTok{)}

\FunctionTok{legend}\NormalTok{(}\StringTok{"topright"}\NormalTok{,}
       \AttributeTok{legend =} \FunctionTok{levels}\NormalTok{(iris}\SpecialCharTok{$}\NormalTok{Species),}
       \AttributeTok{col =}\NormalTok{ colors,}
       \AttributeTok{pch =}\NormalTok{ symbols,}
       \AttributeTok{title =} \StringTok{"Species"}\NormalTok{)}
\end{Highlighting}
\end{Shaded}

\pandocbounded{\includegraphics[keepaspectratio]{RWorksheet_sayson-4b_files/figure-latex/unnamed-chunk-14-1.pdf}}

\begin{Shaded}
\begin{Highlighting}[]
\CommentTok{\#f}
\CommentTok{\#The scatterplot shows that the virginica iris has the overall longest sepal and the setosa iris has the widest sepal with versicolor between them. }
\end{Highlighting}
\end{Shaded}

\begin{Shaded}
\begin{Highlighting}[]
\CommentTok{\#7}
\FunctionTok{library}\NormalTok{(readxl)}

\NormalTok{alexa }\OtherTok{\textless{}{-}} \FunctionTok{read\_excel}\NormalTok{(}\StringTok{"alexa{-}file.xlsx"}\NormalTok{)}

\FunctionTok{print}\NormalTok{(alexa)}
\end{Highlighting}
\end{Shaded}

\begin{verbatim}
## # A tibble: 5 x 5
##   rating date       variation    verified_reviews feedback
##    <dbl> <chr>      <chr>        <chr>               <dbl>
## 1      5 2018-07-30 Black  Dot   Good                    1
## 2      5 2018-07-30 Black Dot    Pretty OK               1
## 3      5 2018-07-30 Black   Dot  Great                   1
## 4      5 2018-07-30 White  Dot   Lackluster              1
## 5      5 2018-07-30 White   Plus NA                      1
\end{verbatim}

\begin{Shaded}
\begin{Highlighting}[]
\CommentTok{\#a}

\NormalTok{alexa}\SpecialCharTok{$}\NormalTok{variation }\OtherTok{\textless{}{-}} \FunctionTok{gsub}\NormalTok{(}\StringTok{"}\SpecialCharTok{\textbackslash{}\textbackslash{}}\StringTok{s+"}\NormalTok{, }\StringTok{" "}\NormalTok{, }\FunctionTok{trimws}\NormalTok{(alexa}\SpecialCharTok{$}\NormalTok{variation))}

\NormalTok{alexa}\SpecialCharTok{$}\NormalTok{variation }\OtherTok{\textless{}{-}} \FunctionTok{gsub}\NormalTok{(}\StringTok{"Black Dot"}\NormalTok{, }\StringTok{"Black"}\NormalTok{, alexa}\SpecialCharTok{$}\NormalTok{variation)}
\NormalTok{alexa}\SpecialCharTok{$}\NormalTok{variation }\OtherTok{\textless{}{-}} \FunctionTok{gsub}\NormalTok{(}\StringTok{"White Dot"}\NormalTok{, }\StringTok{"White"}\NormalTok{, alexa}\SpecialCharTok{$}\NormalTok{variation)}
\NormalTok{alexa}\SpecialCharTok{$}\NormalTok{variation }\OtherTok{\textless{}{-}} \FunctionTok{gsub}\NormalTok{(}\StringTok{"White Plus"}\NormalTok{, }\StringTok{"White Plus"}\NormalTok{, alexa}\SpecialCharTok{$}\NormalTok{variation) }

\NormalTok{knitr}\SpecialCharTok{::}\FunctionTok{include\_graphics}\NormalTok{(}\StringTok{"Purple{-}iris{-}flower{-}plant.png"}\NormalTok{)}
\end{Highlighting}
\end{Shaded}

\pandocbounded{\includegraphics[keepaspectratio]{Purple-iris-flower-plant.png}}

\begin{Shaded}
\begin{Highlighting}[]
\CommentTok{\#b}
\FunctionTok{library}\NormalTok{(dplyr)}
\end{Highlighting}
\end{Shaded}

\begin{verbatim}
## 
## Attaching package: 'dplyr'
\end{verbatim}

\begin{verbatim}
## The following objects are masked from 'package:stats':
## 
##     filter, lag
\end{verbatim}

\begin{verbatim}
## The following objects are masked from 'package:base':
## 
##     intersect, setdiff, setequal, union
\end{verbatim}

\begin{Shaded}
\begin{Highlighting}[]
\NormalTok{variations }\OtherTok{\textless{}{-}}\NormalTok{ alexa }\SpecialCharTok{\%\textgreater{}\%}
\FunctionTok{count}\NormalTok{(variation)}
\FunctionTok{save}\NormalTok{(variations, }\AttributeTok{file =} \StringTok{"variations.RData"}\NormalTok{)}
\NormalTok{variations}
\end{Highlighting}
\end{Shaded}

\begin{verbatim}
## # A tibble: 3 x 2
##   variation      n
##   <chr>      <int>
## 1 Black          3
## 2 White          1
## 3 White Plus     1
\end{verbatim}

\begin{Shaded}
\begin{Highlighting}[]
\CommentTok{\#c}
\NormalTok{colors }\OtherTok{\textless{}{-}} \FunctionTok{c}\NormalTok{(}\StringTok{"Black"} \OtherTok{=} \StringTok{"black"}\NormalTok{, }\StringTok{"White"} \OtherTok{=} \StringTok{"white"}\NormalTok{, }\StringTok{"White Plus"} \OtherTok{=} \StringTok{"lightblue"}\NormalTok{)}
\FunctionTok{barplot}\NormalTok{(}
  \AttributeTok{height =}\NormalTok{ variations}\SpecialCharTok{$}\NormalTok{n,}
  \AttributeTok{names.arg =}\NormalTok{ variations}\SpecialCharTok{$}\NormalTok{variation,}
  \AttributeTok{main =} \StringTok{"Count of Alexa Device Variations"}\NormalTok{,}
  \AttributeTok{xlab =} \StringTok{"Variation"}\NormalTok{,}
  \AttributeTok{ylab =} \StringTok{"Frequency"}\NormalTok{,}
  \AttributeTok{col =}\NormalTok{ colors[variations}\SpecialCharTok{$}\NormalTok{variation],}
  \AttributeTok{las =} \DecValTok{2}
\NormalTok{)}
\FunctionTok{legend}\NormalTok{(}
  \StringTok{"topright"}\NormalTok{,}
  \AttributeTok{legend =}\NormalTok{ variations}\SpecialCharTok{$}\NormalTok{variation,}
  \AttributeTok{fill =}\NormalTok{ colors[variations}\SpecialCharTok{$}\NormalTok{variation],}
  \AttributeTok{title =} \StringTok{"Variation"}
\NormalTok{)}
\end{Highlighting}
\end{Shaded}

\pandocbounded{\includegraphics[keepaspectratio]{RWorksheet_sayson-4b_files/figure-latex/unnamed-chunk-19-1.pdf}}

\begin{Shaded}
\begin{Highlighting}[]
\CommentTok{\#7}

\FunctionTok{load}\NormalTok{(}\StringTok{"variations.RData"}\NormalTok{)}

\NormalTok{bw }\OtherTok{\textless{}{-}}\NormalTok{ variations[variations}\SpecialCharTok{$}\NormalTok{variation }\SpecialCharTok{\%in\%} \FunctionTok{c}\NormalTok{(}\StringTok{"Black"}\NormalTok{, }\StringTok{"White"}\NormalTok{), ]}

\FunctionTok{barplot}\NormalTok{(}
  \AttributeTok{height =}\NormalTok{ bw}\SpecialCharTok{$}\NormalTok{n,}
  \AttributeTok{names.arg =}\NormalTok{ bw}\SpecialCharTok{$}\NormalTok{variation,}
  \AttributeTok{main =} \StringTok{"Comparison of Black and White Variations"}\NormalTok{,}
  \AttributeTok{xlab =} \StringTok{"Variation"}\NormalTok{,}
  \AttributeTok{ylab =} \StringTok{"Count"}\NormalTok{,}
  \AttributeTok{col =} \FunctionTok{c}\NormalTok{(}\StringTok{"black"}\NormalTok{, }\StringTok{"white"}\NormalTok{),}
  \AttributeTok{beside =} \ConstantTok{TRUE}
\NormalTok{)}
\FunctionTok{legend}\NormalTok{(}
  \StringTok{"topright"}\NormalTok{,}
  \AttributeTok{legend =}\NormalTok{ bw}\SpecialCharTok{$}\NormalTok{variation,}
  \AttributeTok{fill =} \FunctionTok{c}\NormalTok{(}\StringTok{"black"}\NormalTok{, }\StringTok{"white"}\NormalTok{),}
  \AttributeTok{title =} \StringTok{"Variation"}
\NormalTok{)}
\end{Highlighting}
\end{Shaded}

\pandocbounded{\includegraphics[keepaspectratio]{RWorksheet_sayson-4b_files/figure-latex/unnamed-chunk-20-1.pdf}}

\end{document}
